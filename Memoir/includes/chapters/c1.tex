\section{ Introduction}
	\paragraph{}
	Scientific events are becoming more and more frequent, which makes their posting on the web a very delicate task, especially with a large number of web pages available on the internet. Therefore, it is important to know how to signal them and make them easy to access, especially interactive and informative for the users.
	
	For the university website, there are no web pages specialised in this type of conference, and it can be seen that those in charge of these conferences always resort to external websites to publish and manipulate the information concerning the conference, and subsequently lose some important clues for the ranking of the university website.
	
	Based on this principle, we want to create a website that takes care of announcing and managing conferences, as well as managing the registrants whether they are accepted or not in the conference.
	
	The implementation will be made as a template to make possible the access and the modification of the conferences and can ask for a session to manage its conference which will be under total control of its creator who is the teacher.
	
	The control of the submission dates of the articles will be under the control of the creator of the conference, each submission will be signalled by an e-mail to the person in charge informing him/her of the submission, the candidate will know the result of his/her submission also by e-mail.
	A list of successful applicants will be made and validated only after payment of the conference fee. Once an applicant has agreed to pay the conference fee, the conference organizer will validate the applicant and send him/her a confirmation message, once the closing date is over, no further submission or payment will be made.
\section{Problematic}
	\paragraph{}
	Most of all universities of the world are publishing their conferences on custom web pages for each one, or they collect them in one website like 10times.com and ieee.org. Meanwhile, any scholar how to want to publish his paper or submit it to a conference, utilize another different website like easychair.org. Our work is related to finding a solution to the problem raised by answering the following questions:
	\begin{itemize}
		\item Why not they all publish everything from conferences to papers in one place?
		\item How to let scholars and universities contact each other from one place?
		\item What is the best solution for this problem and how to achieve that?
	\end{itemize}

\section{Hypotheses}
	\paragraph{}
	Among the proposed solutions, we find that one of them relies on creating a website for publishing conferences and requesting registration in them, where the person in charge of the conference (university or organization) publishes the necessary information about the conference such as its name, date, and participation price... while any researcher or student can request to participate in it, as he sends his research to the officials in charge of the conference and is waiting for it to be accepted by them.
	
\section{Memory structure}
	\paragraph{}
	This article contains the following chapters:
	\begin{itemize}
	
		\item \textbf{Presentation of the project frameworks :}
		Chapter to figure out the problem and its solution in details.
		
		\item \textbf{Analysis and design \textit{(UML)} :}
		Chapter to introduce the \textit{UML} diagrams that we used to analysis the project and figure out his functions.
		
		\item \textbf{The implementation :}
		Chapter to view the technologies that we used in making the site, and the implementation of our site (pictures from the website itself).
	
	\end{itemize}