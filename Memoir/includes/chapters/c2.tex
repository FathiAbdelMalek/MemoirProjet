\section{Presentation of the project frameworks}
	\subsection{Project idea}
		\paragraph{}
		Our project is based on a website of representing conferences and submission to them, when the organizer of the conference put information about his conference like title, description... And any one can see this conference in the web site, but only registered scholars and researchers can submit to it via sending their information to this organizer and wait to his acceptance and confirmation.
	\subsection{Main reason of the project}
		\paragraph{}
		The main reason is that there is no website to represent a conference and let scholars submit in it at the same time, so the conference is in a separated website like ieee.org or in a specific domain like icrami.faox.dk. Meanwhile the scholar publish his paper in another website like easychair.org, so there is no platform in the network that let them all publish everything in the same place.
	\subsection{Project creation}
		\paragraph{}
		We want to create a website that let any organizer of a conference to publish his information like the title, important dates (conference date, submission deadline, confirmation deadline, payment deadline), place, price... etc. Meanwhile any scholar or researcher can submit to it directly from the same web site, by sending to the organizer his information (name, email to contact, abstract of his work, and maybe the work itself, the authors... etc). but first they should be registered in this site.
		\paragraph{}
		There is some rules to submit, the scholar should demand a submission in a conference before the end of his submission deadline, once the its end, he couldn't submit, update or delete his submissions for that conference. The organizer can't accept it immediately, he should wait until the beginning of confirmation deadline, but he can see and read it at any time he want, when he accept a scholar, this scholar will get a notification and an email to paid for the submission before end of payment deadline, or he will pay a larger price, he take a screenshot of the payment ticket and send it to the organizer and wait his confirmation.
		\paragraph{}
		To publish a conference or submit to it, the user should be registered in the site, he register with the email, user-name, first and last name at the beginning, but he can complete the other information later via access his profile page and press (edit), he get a form to add his (birth date, sex, country, work place, degree, specialty, website if any and an profile image), all this information are optionally, but the organizer can either accept or refuse the submission depends on his rules. When a user submit to a conference he put the information below (his first and last name, his email, name of the paper, an abstract of his work, the paper itself and the authors of it), all this information are required expect the article itself, it depends on the conference rules, once the organizer read it and the confirmation deadline begin, he can accept or refuse it at any time, in case of acceptance, the submitter receive a notification in the site and an email of acceptance, and he should pay for the conference and send the payment ticket to the organizer who will confirm his submission after receiving it.
		\paragraph{}
		The conference organizer should put many information in the description text area including his payment method, schedule of the conference and many other information.
		
		